\documentclass{article}
\usepackage{graphicx}


\title{The Effect of Alanine on Spore Growth:Part 1 of 2 }
\date{February 13, 2020}
 
\begin{document}

\maketitle
 
This report present the measurement for each well in a 96 well plate. The experimental variables were the initial concentration of spores (from minimal, quantified as 0.1 mln spores, to 100 mln spores). and alanine concentration (either 0 or 1.0 mM). 

The study ran for over 24 hours. The recorded temperature of the plate reader was between 29.9 and 31.1 during the study. 

The addition of alanine significantly increased the RFU readings, even before the log  growth phase, which varied well-to-well, but roughly occurred from 12 - 20 hours. Some wells never moved into a log phase and RFU stayed a minimal and constant during the study period. 

If deemed relevant, the growth coefficient for each well can be calculated as an index for comparison between the spore count and alanine concentration. 

Each individual well readings are presented (row 1 of each figure). Then the same reading presented compared to the well's experimental group (row 2 of each figure). Then the individual well and the experimental group are compared to the same spore count, but with or without Alanine additives. 

Can the growth coefficient be calculated by assumming exponential growth?

\[ y =C e^{xt}\]

where \textit{y}\  is the microplate measurement in RFU, \textit{C}\  is the minimum measurement, \textit{x}\  is the exponential growth coefficient in 1/hours, and \textit{t}\  is the time lapsed in hours since the beginning of the experiment. The growth coefficient would be solved by averaging the growth coefficient during the log phase growth (from roughly hour 12 - hour 20). 

\[x=  \ln{  \frac{y}{C} }  * 1/t\]



\begin{figure}[!hb]
\centering
\includegraphics[scale = 0.5]{ A3.png}
\end{figure}

\begin{figure}[!hb]
\centering
\includegraphics[scale = 0.5]{ A4.png}
\end{figure}

\begin{figure}[!hb]
\centering
\includegraphics[scale = 0.5]{ A5.png}
\end{figure}

\begin{figure}[!hb]
\centering
\includegraphics[scale = 0.5]{ A6.png}
\end{figure}

\begin{figure}[!hb]
\centering
\includegraphics[scale = 0.5]{ A7.png}
\end{figure}

\begin{figure}[!hb]
\centering
\includegraphics[scale = 0.5]{ A8.png}
\end{figure}

\begin{figure}[!hb]
\centering
\includegraphics[scale = 0.5]{ A9.png}
\end{figure}

\begin{figure}[!hb]
\centering
\includegraphics[scale = 0.5]{ A10.png}
\end{figure}

\begin{figure}[!hb]
\centering
\includegraphics[scale = 0.5]{ A11.png}
\end{figure}

\begin{figure}[!hb]
\centering
\includegraphics[scale = 0.5]{ A12.png}
\end{figure}

\begin{figure}[!hb]
\centering
\includegraphics[scale = 0.5]{ B1.png}
\end{figure}

\begin{figure}[!hb]
\centering
\includegraphics[scale = 0.5]{ B2.png}
\end{figure}

\begin{figure}[!hb]
\centering
\includegraphics[scale = 0.5]{ B3.png}
\end{figure}

\begin{figure}[!hb]
\centering
\includegraphics[scale = 0.5]{ B4.png}
\end{figure}

\begin{figure}[!hb]
\centering
\includegraphics[scale = 0.5]{ B5.png}
\end{figure}

\begin{figure}[!hb]
\centering
\includegraphics[scale = 0.5]{ B6.png}
\end{figure}


\begin{figure}[!hb]
\centering
\includegraphics[scale = 0.5]{ B7.png}
\end{figure}

\begin{figure}[!hb]
\centering
\includegraphics[scale = 0.5]{ B8.png}
\end{figure}

\begin{figure}[!hb]
\centering
\includegraphics[scale = 0.5]{ B9.png}
\end{figure}

\begin{figure}[!hb]
\centering
\includegraphics[scale = 0.5]{ B10.png}
\end{figure}

\begin{figure}[!hb]
\centering
\includegraphics[scale = 0.5]{ B11.png}
\end{figure}

\begin{figure}[!hb]
\centering
\includegraphics[scale = 0.5]{ B12.png}
\end{figure}

\begin{figure}[!hb]
\centering
\includegraphics[scale = 0.5]{ C1.png}
\end{figure}

\begin{figure}[!hb]
\centering
\includegraphics[scale = 0.5]{ C2.png}
\end{figure}

\begin{figure}[!hb]
\centering
\includegraphics[scale = 0.5]{ C3.png}
\end{figure}

\begin{figure}[!hb]
\centering
\includegraphics[scale = 0.5]{ C4.png}
\end{figure}

\begin{figure}[!hb]
\centering
\includegraphics[scale = 0.5]{ C5.png}
\end{figure}

\begin{figure}[!hb]
\centering
\includegraphics[scale = 0.5]{ C6.png}
\end{figure}

\begin{figure}[!hb]
\centering
\includegraphics[scale = 0.5]{ C7.png}
\end{figure}

\begin{figure}[!hb]
\centering
\includegraphics[scale = 0.5]{ C8.png}
\end{figure}

\begin{figure}[!hb]
\centering
\includegraphics[scale = 0.5]{ C9.png}
\end{figure}

\begin{figure}[!hb]
\centering
\includegraphics[scale = 0.5]{ C10.png}
\end{figure}

\begin{figure}[!hb]
\centering
\includegraphics[scale = 0.5]{ C11.png}
\end{figure}

\begin{figure}[!hb]
\centering
\includegraphics[scale = 0.5]{ C12.png}
\end{figure}


\begin{figure}[!hb]
\centering
\includegraphics[scale = 0.5]{ D1.png}
\end{figure}

\begin{figure}[!hb]
\centering
\includegraphics[scale = 0.5]{ D2.png}
\end{figure}

\begin{figure}[!hb]
\centering
\includegraphics[scale = 0.5]{ D3.png}
\end{figure}

\begin{figure}[!hb]
\centering
\includegraphics[scale = 0.5]{ D4.png}
\end{figure}

\begin{figure}[!hb]
\centering
\includegraphics[scale = 0.5]{ D5.png}
\end{figure}

\begin{figure}[!hb]
\centering
\includegraphics[scale = 0.5]{ D6.png}
\end{figure}

\begin{figure}[!hb]
\centering
\includegraphics[scale = 0.5]{ D7.png}
\end{figure}

\begin{figure}[!hb]
\centering
\includegraphics[scale = 0.5]{ D8.png}
\end{figure}

\begin{figure}[!hb]
\centering
\includegraphics[scale = 0.5]{ D9.png}
\end{figure}

\begin{figure}[!hb]
\centering
\includegraphics[scale = 0.5]{ D10.png}
\end{figure}

\begin{figure}[!hb]
\centering
\includegraphics[scale = 0.5]{ D11.png}
\end{figure}

\begin{figure}[!hb]
\centering
\includegraphics[scale = 0.5]{ D12.png}
\end{figure}


\end{document}

]