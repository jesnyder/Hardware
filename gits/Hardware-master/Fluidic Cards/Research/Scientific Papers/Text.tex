\documentclass[a4paper, 11pt]{article}
\usepackage[left=3cm, right=3cm, top=2cm]{geometry}

\usepackage[numbers]{natbib}
\bibliographystyle{unsrtnat}
\usepackage[document]{ragged2e}

\setcounter{tocdepth}{1}  % Set the table of contents depth to one header

\begin{document}


\author{Jessica E. Snyder jessica.e.snyder@nasa.gov}
\title{ Research}
\maketitle

\tableofcontents
%\newpage


\section{Optimizing gas permeability of a PDMS membrane}

Cross-linking additives (10\% vinyl groups and tetraethoxysilicone) increased gas permeability (oxygen and nitrogen) above theoretical limits of PDMS  \cite{rao2007preparation}

Thickness must be more than tens of micrometers, up to 1 mm. Above 50 um, the permeability across a PDMS membrane is independent of its thickness. For membranes more than 50 nm and less than 50um, gas permeability decreases with membrane thickness until 50nm \cite{firpo2015permeability}. Then defect in the PDMS membrane dominate the transport mechanism. 

\section{Ductile PDMS substrates}
Curing temperature closer to 25 deg C produce more ductile substrates \cite{johnston2014mechanical}. 

\section{Freeze-thaw cycling of PDMS substrates}
PDMS service temperature is -45 to 200 deg C \cite{Slygard184ProductSheet}.

\section{Manufacturing method}

The goal is an array of shallow wells embedded in a PDMS substrate. 

(1) Mix the PDMS (10:1 by weight) - 33 mL in a centrifuge tube. 
(2) Invert the tube 50 - 100 times to mix
(3) Centrifuge 11000 rpm for 15 minutes to removes bubbles 
(4) Pour 10 mL into dish
(5) Let settle then place on low heat 
(6) Place template on top of the cured PDMS, weight with 

\bibliographystyle{plainnat} 
\bibliography{References}
\end{document}






